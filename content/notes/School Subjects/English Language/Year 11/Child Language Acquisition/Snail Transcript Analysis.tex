% Options for packages loaded elsewhere
\PassOptionsToPackage{unicode}{hyperref}
\PassOptionsToPackage{hyphens}{url}
%
\documentclass[
]{article}
\usepackage{amsmath,amssymb}
\usepackage{iftex}
\ifPDFTeX
  \usepackage[T1]{fontenc}
  \usepackage[utf8]{inputenc}
  \usepackage{textcomp} % provide euro and other symbols
\else % if luatex or xetex
  \usepackage{unicode-math} % this also loads fontspec
  \defaultfontfeatures{Scale=MatchLowercase}
  \defaultfontfeatures[\rmfamily]{Ligatures=TeX,Scale=1}
\fi
\usepackage{lmodern}
\ifPDFTeX\else
  % xetex/luatex font selection
\fi
% Use upquote if available, for straight quotes in verbatim environments
\IfFileExists{upquote.sty}{\usepackage{upquote}}{}
\IfFileExists{microtype.sty}{% use microtype if available
  \usepackage[]{microtype}
  \UseMicrotypeSet[protrusion]{basicmath} % disable protrusion for tt fonts
}{}
\makeatletter
\@ifundefined{KOMAClassName}{% if non-KOMA class
  \IfFileExists{parskip.sty}{%
    \usepackage{parskip}
  }{% else
    \setlength{\parindent}{0pt}
    \setlength{\parskip}{6pt plus 2pt minus 1pt}}
}{% if KOMA class
  \KOMAoptions{parskip=half}}
\makeatother
\usepackage{xcolor}
\setlength{\emergencystretch}{3em} % prevent overfull lines
\providecommand{\tightlist}{%
  \setlength{\itemsep}{0pt}\setlength{\parskip}{0pt}}
\setcounter{secnumdepth}{-\maxdimen} % remove section numbering
\ifLuaTeX
  \usepackage{selnolig}  % disable illegal ligatures
\fi
\IfFileExists{bookmark.sty}{\usepackage{bookmark}}{\usepackage{hyperref}}
\IfFileExists{xurl.sty}{\usepackage{xurl}}{} % add URL line breaks if available
\urlstyle{same}
\hypersetup{
  pdftitle={Snail Transcript Presentation},
  pdfauthor={Garv Shah},
  hidelinks,
  pdfcreator={LaTeX via pandoc}}

\title{Snail Transcript Presentation}
\author{Garv Shah}
\date{2022-05-22}

\begin{document}
\maketitle

\hypertarget{introduction}{%
\subsection{Introduction}\label{introduction}}

\begin{itemize}
\item
  Conversation between Bella, girl of 1 year 11 months, and her
  Grandmother in the garden about snails
\item
  Grandmother is her caregiver, serves as MKO while helping Bella, the
  child, in her language development.
\item
  Throughout presentation, red C for child, blue G for grandma
\item
  Bella well into the telegraphic stage, evidence provider later
\item
  Subsystems developed into that expected of 2-3 year old, about 5
  months ahead of expected language development
\item
  ``We are storytelling creatures, and as children we acquire language
  to tell those stories that we have inside us.'' - Bruner, key theorist
  behind Interactionism
\item
  Bella learns how to describe the world around her by imitating and
  interacting with her MKO, grandma
\end{itemize}

\hypertarget{features-of-language}{%
\subsection{Features of Language}\label{features-of-language}}

\hypertarget{emerging-subsystems}{%
\subsubsection{Emerging Subsystems}\label{emerging-subsystems}}

aka Bella\textquotesingle s Developmental Stage

\begin{itemize}
\item
  Well into the telegraphic stage
\item
  At this point, Bella gone beyond two-word stage , producing coherent
  enough utterances, but somewhat lacking all the needed function words
  and morphemes to be syntactically accurate
\item
  Children in the telegraphic stage almost sounds like text messages,
  omitting unnecessary words.
\item
  Also gained the ability to ask basic questions, like "where" or "why"
\item
  For example, child might say "Where daddy?" or "What that?" dropping
  the word *is
\end{itemize}

\textbf{Lexical and Semantic Perspective}:

\begin{itemize}
\tightlist
\item
  Actively asking where questions: "Where {[}nʌdə sneɪjəl{]}"
\item
  Can point and direct others: "Look! I see {[}ənʌdə sneɪjəw{]}"
\end{itemize}

\textbf{Morphological \& Syntactic Perspective}

\begin{itemize}
\item
  We can clearly see Bella well into stage as she is beginning to add
  grammatical morphemes to speech:

  \begin{itemize}
  \tightlist
  \item
    "{[}dɛəz{]} Mickey" uses contraction for "there is"
  \item
    "Look he {[}pʊdɪn{]} his head way up in sky": still developing
    inflectional morpheme -ing, g-dropping: replacing /ŋ/ sound with
    /n/, putting -\textgreater{} puttin\textquotesingle{}
  \end{itemize}
\item
  That last utterance was outlier, MLU is approx 3.44 lexemes, while
  this utterance had 9 lexemes, much higher than mean
\item
  This is a common phrase, almost a collocation, possibly replacing the
  word sky with "clouds" or "stars".
\item
  Common phrase could have been repeated by MKO such as parents or
  grandmother and imitated by Bella, supporting
  Skinner\textquotesingle s Behaviourist ideas
\end{itemize}

Nonetheless, clear that Bella well into telegraphic stage, with good
syntactic knowledge and understanding of location, and addition of the
contractive "is" and gerund as part of speech.\\
Being said, still does not use many conjunctions or why questions, place
approximately 5 months ahead of expected language development.

\hypertarget{supported-theories}{%
\subsubsection{Supported Theories}\label{supported-theories}}

aka Carer Strategies

\begin{itemize}
\item
  Mainly supports behaviourism + interactionism, evidently learning
  through interaction with MKO
\item
  Lines that support cognitivism are present

  \begin{itemize}
  \tightlist
  \item
    Displays a proper understanding of location:

    \begin{itemize}
    \tightlist
    \item
      G: "Do you see another one?"
    \item
      C: "{[}ʌn də flaʊwə{]}"
    \end{itemize}
  \item
    She would not be able to talk about location of snail if she
    didn\textquotesingle t understand location as concept, so
    cognitivism is supported in this way

    \begin{itemize}
    \tightlist
    \item
      G: "Do you see another snail?"
    \item
      C: "{[}ən dæ twiː{]}"
    \end{itemize}
  \item
    Bella has also begun to use simple prepositions, such as "on" or
    "in"
  \end{itemize}
\item
  Not much evidence of innateness ∵ lack of ``virtuous errors'', active
  role played by Grandma
\item
  Throughout transcription, Child-directed-speech has constant
  repetition of phrases more "correctly" by MKO, evidence of operant
  conditioning and scaffolding.

  \begin{itemize}
  \tightlist
  \item
    eg. "Look! I see {[}ənʌdə sneɪjəw{]}", Grandma replied back, "Do you
    see another snail?"
  \item
    Repetition of more correct utterance supports the idea that Bella
    will imitate, -\textgreater{} negative reinforcement, supports
    behaviourist ideas
  \item
    The transition of the declarative sentence to the interrogative
    sentence shows how MKO is scaffolding for child, expanding ZPD by
    questioning -\textgreater{} interactionist ideas supported
  \end{itemize}
\end{itemize}

\hypertarget{interactionism}{%
\paragraph{Interactionism}\label{interactionism}}

Theory emphasises the interaction between children and their caregivers,
in this case Bella \& Grandma\\
Focuses a lot on ZPD and scaffolded needed for learning to develop

\begin{itemize}
\tightlist
\item
  G: "Yes he\textquotesingle s in the tree" later Bella says "Look he
  {[}pʊdɪn{]}"
\item
  Strongly supports interactionist theory: Bella no way of knowing the
  snail is male, but bc Grandma automatically assumes gender, Bella
  learns to assume the same. As such, Bella learns about
  society\textquotesingle s perceptions of the world
\end{itemize}

\hypertarget{behaviourism}{%
\paragraph{Behaviourism}\label{behaviourism}}

\begin{itemize}
\tightlist
\item
  Supported by various examples of recasts throughout transcript, uses
  operant conditioning for negative reinforcement
\item
  Prime example is

  \begin{itemize}
  \tightlist
  \item
    C: "Where {[}nʌdə sneɪjəl{]}"
  \item
    G: "Where\textquotesingle s another snail"
  \end{itemize}
\item
  Later in exact same passage, Bella is seen correcting herself by
  imitating her Grandma, repeating the exact same recast in a more
  phonetically correct lexeme "another" (not ellided)

  \begin{itemize}
  \tightlist
  \item
    G: "There\textquotesingle s another snail"
  \item
    C: "{[}dɛəz ənʌdə sneɪjəw{]}"
  \end{itemize}
\item
  This recast also supports interactionism, as it provides scaffolding
  to build on ZPD
\end{itemize}

\hypertarget{the-subsystems}{%
\subsection{The Subsystems}\label{the-subsystems}}

\hypertarget{phonological-processes}{%
\subsubsection{Phonological Processes}\label{phonological-processes}}

Many phonological processes taking place in Bella\textquotesingle s
language, and though she\textquotesingle s mostly intelligible, clear
that her phonetic ability is at that of telegraphic stage

\begin{itemize}
\tightlist
\item
  ``{[}kaɪm{]} on {[}tiː{]} a {[}sneɪjəl{]}, see''

  \begin{itemize}
  \tightlist
  \item
    Elision of the /l/ and /ɹ/ consonant clusters -\textgreater{}
    cluster reduction
  \item
    Epenthesis of the /jə/ sound
  \end{itemize}
\item
  "{[}dɛəz ənʌdə sneɪjəw{]}"

  \begin{itemize}
  \tightlist
  \item
    Can see that epenthesis of /jə/ sound is consistent across her
    speech
  \item
    /l/ consonant cluster also struggles her with an example of gliding
    the /l/ to /w/
  \end{itemize}
\item
  "Where {[}nʌdə sneɪjəl{]}"\\
  - Can see all examples so far had consistent th-stops turning /ð/
  -\textgreater{} /d/, has a lot of gliding but mostly intelligible,
  indication of child-like speech\\
  - also elides /ə/ sound sometimes, especially when at start of word,
  but corrected by imitation and operant conditioning\\
  Consistent th-stopping makes sense of age group, as children not
  expected to properly articulate /ð/ until 4 - 6 years of age.\\
  Because /ð/ sound is quite complex, have to remember to breathe out
  with tiny gap between teeth and tongue. If this gap is forgotten /d/
  is produced instead. Evidently sounds are very close in mouth, and
  subtle differences not picked up this early, especially at less than 2
  years old.
\end{itemize}

\hypertarget{lexicology}{%
\subsubsection{Lexicology}\label{lexicology}}

\begin{quote}
in ppt make sure to add couple transcribed examples
\end{quote}

Bella\textquotesingle s demonstrated lexicon is about 20 words give or
take, repeating "there\textquotesingle s", "another", "snail", and
"where" quite a bit.\\
The repetition of phrases not unexpected at this development stage.
Phrase "look he puttin\textquotesingle{} his head way up in sky" from
before was well beyond demonstrated lexicon of Bella\textquotesingle s
throughout convo, further show outlier.

\begin{itemize}
\tightlist
\item
  Grandma and Bella probably spend quite bit of time in garden, as most
  content words she knows are under the semantic field of nature, such
  as "climb, tree, snail"
\item
  Also knows words for people she spends lot of time with, such as
  brother "Mickey", supporting interactionist ideas that interaction
  with MKO reinforces language learning\\
  New words are introduced through questioning and scaffolding, once
  again supporting interactionism
\item
  G: "What colour is that flower?"
\item
  C: "Look! I see {[}ənʌdə sneɪjəw{]}"
\item
  Bella is distracted, and does not answer the question, but is still
  prompted to use new lexemes to describe her environment
\end{itemize}

\hypertarget{morphology}{%
\subsubsection{Morphology}\label{morphology}}

As noted before, Bella is able to utilise the gerund, meaning she can
use the inflectional morpheme -ing.

\begin{itemize}
\tightlist
\item
  Contrapuntally, she does not appear to use the -s morpheme to indicate
  the plural of snail, showing that her morphological progress is in the
  early phase of the telegraphic stage\\
  Bella also shows evidence of articles such as "a" appearing, showing
  that she is beginning to use more articles and function words in her
  speech:
\item
  G: Look here\textquotesingle s a snail
\item
  C: "{[}hiːəz~ə sneɪjəl{]}"\\
  Also very clear evidence of a recast, imitating what MKO is saying and
  learning to add more grammar to speech.\\
  Overall, Bella\textquotesingle s morphological ability is at expected
  stage for a two year old, where utterances have a clear hierarchical
  structure but is not yet that of adult grammar.
\end{itemize}

\hypertarget{syntaxsemantics}{%
\subsubsection{Syntax/Semantics}\label{syntaxsemantics}}

\begin{itemize}
\tightlist
\item
  Both syntax and semantics of Bella\textquotesingle s speech have been
  explored previously in presentation
\item
  To reiterate, errors are dealt with by caregiver in a very
  behaviourist fashion, by recasting or repeating the utterance in a
  more correct form
\item
  Semantically, is able to use interrogative sentences to question the
  world around her, using many "where" questions
\item
  Still does not use any compound sentences yet, but SVO structure is
  present, showing it\textquotesingle s developing

  \begin{itemize}
  \tightlist
  \item
    "I see {[}ənʌdə sneɪjəw{]}"
  \item
    Subject "I", Verb "see", Object "another snail"
  \end{itemize}
\end{itemize}

\hypertarget{discourse}{%
\subsubsection{Discourse}\label{discourse}}

Finally, Bella is picking up discourse features well.\\
Adjacency pairs:

\begin{itemize}
\tightlist
\item
  G: "You want me to climb in the tree?"
\item
  C: "{[}jɑː{]}"
\item
  G: "No way, silly monkey! You want to climb in the tree?"
\item
  C: "{[}jɑː{]}"\\
  These adjacency pairs show that Bella is just saying yes to question
  without necessarily knowing the meaning, but she understands that it
  is a question.
\end{itemize}

Bella also mostly coherent, participating in turn taking structure as
seen above. Bella also sometimes uses discourse markers in her speech:

\begin{itemize}
\tightlist
\item
  C: "oh, oh!"\\
  This interjection of surprise conveys meaning to excitement, conveying
  meaning to MKO.
\end{itemize}

\textbf{Caretaker}\\
Caretaker uses confirmation requests to prompt and scaffold the child.

\begin{itemize}
\tightlist
\item
  G: "That\textquotesingle s a lot of snails, isn\textquotesingle t it?"
\item
  or G: "You put the snail in the garden, did you?"\\
  Scaffolding once again supports interactionist theories, as the
  discourse between the child and caregiver driving the conversation
  teaches Bella language.
\end{itemize}

\hypertarget{conclusion}{%
\subsection{Conclusion}\label{conclusion}}

\begin{itemize}
\item
  Overall, Bella is quite relatively for her age. She seems to be well
  into the telegraphic stage of her language development, about 5 months
  ahead of the expected development at her age.
\item
  This is evident by the emerging subsystems evident in her speech, such
  as the ability to ask questions and use prepositions of location.
\item
  Throughout the passage, the grandma acts as a MKO and uses development
  strategies that very closely align with the Behaviourist and
  Interactionist language acquisition theories.
\item
  Bella will likely continue to refine her language use as she
  approaches more adult-like grammatical ability, adding more function
  words and morphemes into her speech.

  \begin{itemize}
  \tightlist
  \item
    She already shows evidence of this, such as the inklings of using
    the inflectional gerund and basic article usage, but will develop
    more
  \end{itemize}
\item
  Like to finish off with a quote

  \begin{quote}
  ``In sum, then, "thinking about thinking" has to be a principal
  ingredient of any empowering practice of education.''
  \end{quote}
\end{itemize}

Thank you for listening!

\end{document}
