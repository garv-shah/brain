%%
% Please see https://bitbucket.org/rivanvx/beamer/wiki/Home for obtaining beamer.
%%
\documentclass[compress]{beamer}
\usetheme{Boadilla}
\usepackage{xcolor}
\usepackage{csquotes}
\usepackage{fontspec}
\useoutertheme{miniframes}

%Beamer notes config%
\usepackage{pgfpages}
\setbeameroption{show notes on second screen}

%Defines the IPA environment and ipatext functions%
\newfontfamily\ipafont{Charis SIL}
\newenvironment{ipa}{\ipafont}{\par}
\DeclareTextFontCommand{\ipatext}{\ipafont}

%Makes all descriptions use the IPA font%
\let\origdescription=\description
\def\description{\origdescription\ipa}

\title{Snail Transcript}
\subtitle{Child Language Aquisition}
\author{Garv Shah}
\institute{English Language}
\date{\today}

\begin{document}

%Titlepage%
\begin{frame}
\titlepage
\end{frame}

%Table of Contents%
\begin{frame}
\frametitle{Outline}

\begin{columns}
\column{0.5\textwidth}
	\tableofcontents

\column{0.4\textwidth}
	\centering
	\begin{figure}
		\includegraphics[scale=0.3]{snail.jpg}
		\caption{Example of a Snail}
	\end{figure}
\end{columns}

\end{frame}

%Introduction%
\section{Introduction}
\begin{frame}
	\frametitle{Introduction}

	\begin{itemize}
		\item Bella and Grandmother talking about snail
	\end{itemize}

	\begin{description}
		\item[\textcolor{red}{C}] Bella, the child learning English
		\item[G] Her grandmother, caregiver and MKO
	\end{description}

	\begin{itemize}
		\item Bella well into the telegraphic stage
		\item About 5 months ahead
	\end{itemize}
\end{frame}

\note[itemize]{
\item Conversation between Bella, girl of 1 year 11 months, and her Grandmother in the garden about snails
\item Grandmother is her caregiver, serves as MKO while helping Bella, the child, in her language development.
\item Throughout presentation, red C for child, blue G for grandma
\item Bella well into the telegraphic stage, evidence provider later
\item Subsystems developed into that expected of 2-3 year old, about 5 months ahead of expected language development
}

\begin{frame}
	\frametitle{Introduction}
	
	\begin{displayquote}
		We are storytelling creatures, and as children we acquire language to tell those stories that we have inside us.” - Jerome Bruner	
	\end{displayquote}

\end{frame}

\note[itemize]{
\item Bella learns how to describe the world around her by imitating and
  interacting with her MKO, grandma
}

\section{Features of Language}
\subsection{Emerging Subsystems}
\begin{frame}
	\frametitle{Emerging Subsystems}
	\begin{itemize}
		\item Bella's Developmental Stage $\rightarrow$ well into telegraphic
		\item Coherent utterances, but missing function words/morphemes
	\end{itemize}
	
	\begin{block}{Typical Utterances}
		\begin{columns}
			\column{0.5\textwidth}
				\centering
				\begin{displayquote}
					"Where daddy?"
				\end{displayquote}

			\column{0.5\textwidth}
				\centering
				\begin{displayquote}
					"What that?
				\end{displayquote}
		\end{columns}
	\end{block}

	
\end{frame}

\note[itemize]{
\item
  Well into the telegraphic stage
\item
  At this point, Bella gone beyond two-word stage , producing coherent
  enough utterances, but somewhat lacking all the needed function words
  and morphemes to be syntactically accurate
\item
  Children in the telegraphic stage almost sounds like text messages,
  omitting unnecessary words.
\item
  Also gained the ability to ask basic questions, like "where" or "why"
\item
  For example, child might say "Where daddy?" or "What that?" dropping
  the word \emph{is}
}

\subsubsection{Lexical/Semantic POV}
%Lexical and Semantic Perspective%
\begin{frame}
	\frametitle{Lexical/Semantic Perspective}
	
	\begin{itemize}
		\item Actively asking where questions
	\end{itemize}
	\begin{description}
	\item[\textcolor{red}{C}] Where [nʌdə sneɪjəl]
	\end{description}
	
	\begin{itemize}
		\item Can point and direct others
	\end{itemize}
	\begin{description}
	\item[\textcolor{red}{C}] Look! I see [ənʌdə sneɪjəw]
	\end{description}
\end{frame}

\note[itemize]{
\item
  Actively asking where questions: "Where \ipa{{[}nʌdə sneɪjəl{]}}"
\item
  Can point and direct others: "Look! I see \ipa{{[}ənʌdə sneɪjəw{]}}"
}

\subsubsection{Morphological/Syntactic POV}
%Morphological & Syntactic Perspective%
\begin{frame}
	\frametitle{Morphological/Syntactic Perspective}
	
	\begin{itemize}
		\item Grammatical morphemes being added to speech
	\end{itemize}
	\begin{description}
		\item[\textcolor{red}{C}] [dɛəz] Mickey
		\item[\textcolor{red}{C}] Look he [pʊdɪn] his head way up in sky
	\end{description}
	
	MLU is 3.44 lexemes, last utterance an outlier
	\begin{itemize}
		\item Could be collocation: "head way up in the clouds" or "head way up in the stars"
		\item Supports Skinner's behaviourist ideas
	\end{itemize}
\end{frame}

\note[itemize]{
\item
  We can clearly see Bella well into stage as she is beginning to add
  grammatical morphemes to speech:

\item
  "\ipa{{[}dɛəz{]}} Mickey" uses contraction for "there is"
\item
  "Look he \ipa{{[}pʊdɪn{]}} his head way up in sky": still developing inflectional morpheme -ing, g-dropping: replacing /ŋ/ sound with /n/, putting $\therefore$ puttin'{}
\item
  That last utterance was outlier, MLU is approx 3.44 lexemes, while
  this utterance had 9 lexemes, much higher than mean
\item
  This is a common phrase, almost a collocation, possibly replacing the
  word sky with "clouds" or "stars".
\item
  Common phrase could have been repeated by MKO such as parents or
  grandmother and imitated by Bella, supporting
  Skinner's Behaviourist ideas
}

\subsection{Supported Theories}
%Carer Strategies%
\begin{frame}
	\frametitle{Supported Theories}
	
	Mainly supports behaviourism and interactionism
	
	Lines that support cognitivism are present
	
	\begin{itemize}
		\item Displays understanding of location
	\end{itemize}
	\begin{description}
		\item[G] Do you see another one?
		\item[\textcolor{red}{C}] [ʌn də flaʊwə]
	\end{description}
	
	\begin{itemize}
		\item Simple prepositions such as "on" or "in"
	\end{itemize}
	\begin{description}
		\item[G] Do you see another snail?
		\item[\textcolor{red}{C}] [ən dæ twiː]
	\end{description}
\end{frame}

\begin{frame}
	\frametitle{Supported Theories}
	
	\begin{itemize}
		\item Recast $\therefore$ negative reinforcement
	\end{itemize}
	\begin{description}
		\item[\textcolor{red}{C}] Look! I see [ənʌdə sneɪjəw]
		\item[G] Do you see another snail
	\end{description}
	
	\begin{block}{}
		Transition of declarative $\rightarrow$ interrogative $=$ scaffolding
	\end{block}
\end{frame}

%Interactionism%
\subsubsection{Interactionism}
\begin{frame}
	\frametitle{Interactionism}
	
	Emphasis on interaction with MKO
	
	\begin{itemize}
		\item Teaches society's perceptions through interaction
	\end{itemize}
	\begin{description}
		\item[G] Yes he's in the tree
		\item[\textcolor{red}{C}] Look he [pʊdɪn]...
	\end{description}
\end{frame}

%Behaviourism%
\subsubsection{Behaviourism}
\begin{frame}
	\frametitle{Behaviourism}
	
	\begin{block}{}
		Operant conditioning through recasts, Bella learns to imitate MKO
	\end{block}
	
	\noindent\rule{\textwidth}{1pt}
	
	\begin{description}
		\item[\textcolor{red}{C}] Where [nʌdə sneɪjəl]
		\item[G] Where's another snail
	\end{description}
	
	\noindent\rule{\textwidth}{1pt}
	
	\begin{description}
		\item[G] There's another snail
		\item[\textcolor{red}{C}] [dɛəz ənʌdə sneɪjəw]
	\end{description}
	
\end{frame}

\section{The Subsystems}
\begin{frame}
	\vfill
	\centering
	\begin{beamercolorbox}[sep=8pt,center,shadow=true,rounded=true]{title}
	\usebeamerfont{title}\insertsectionhead\par%
 	\end{beamercolorbox}
 	\vfill
 \end{frame}

\subsection{Phonological Processes}
%Phonological Processes%
\begin{frame}
	\frametitle{Phonological Processes}
	
	Operant conditioning through recasts, Bella learns to imitate MKO
	
	\begin{itemize}
		\item Elision of the /l/ and /ɹ/ consonant clusters $\rightarrow$ cluster reduction
		\item Epenthesis of the /jə/ sound
	\end{itemize}
	\begin{description}
		\item[\textcolor{red}{C}] [kaɪm] on [tiː] a [sneɪjəl], see
	\end{description}
	
	\begin{itemize}
		\item Epenthesis of /jə/ sound is consistent across her speech
		\item Gliding of /l/ $\rightarrow$ /w/
	\end{itemize}
	\begin{description}
		\item[\textcolor{red}{C}] [dɛəz ənʌdə sneɪjəw]
	\end{description}
	
	\begin{itemize}
		\item Consistent th-stopping
	\end{itemize}
	\begin{description}
		\item[\textcolor{red}{C}] Where [nʌdə sneɪjəl]
	\end{description}
	
\end{frame}

\subsection{Lexicology}
%Lexicology%
\begin{frame}
	\frametitle{Lexicology}
	
	Lexicon of $\approx$ 20 words
	
	\begin{itemize}
		\item Repeating lexemes: "there's", "another", "snail", "where"
		\item Probably spends a lot of time in the garden
		\item Supporting interactionist theories
	\end{itemize}
	
	\begin{itemize}
		\item New words introduced through questioning and scaffolding, prompted for new lexemes
	\end{itemize}
	\begin{description}
		\item[G] What colour is that flower?
		\item[\textcolor{red}{C}] Look! I see [ənʌdə sneɪjəw]
	\end{description}
	
\end{frame}

\subsection{Morphology}
%Morphology%
\begin{frame}
	\frametitle{Morphology}
	
	Able to utilise gerund and starting to use inflectional morphemes
	
	\begin{itemize}
		\item Evidence of using articles
		\item Clear example of learning to add grammar from recast
	\end{itemize}
	\begin{description}
		\item[G] Look here's a snail
		\item[\textcolor{red}{C}] [hiːəz ə sneɪjəl]
	\end{description}
	
\end{frame}

\subsection{Syntax/Semantics}
%Syntax/Semantics%
\begin{frame}
	\frametitle{Syntax/Semantics}
	
	Errors dealt with in behaviourist fashion
	Is able to use interrogative sentences
	
	\begin{itemize}
		\item SVO structure present
	\end{itemize}
	\begin{description}
		\item[\textcolor{red}{C}] \textcolor{olive}{I} \textcolor{teal}{see} \textcolor{gray}{[ənʌdə sneɪjəw]}
	\end{description}
	
	\begin{block}{Key}
	\begin{columns}
	\column{0.3\textwidth}
		\centering
		\textcolor{olive}{Subject}

	\column{0.3\textwidth}
		\centering
		\textcolor{teal}{Verb}
		
	\column{0.3\textwidth}
		\centering
		\textcolor{gray}{Object}
	\end{columns} 
	\end{block}
	
\end{frame}

\subsection{Discourse}
%Discourse%
\begin{frame}
	\frametitle{Discourse}
	
	\begin{itemize}
		\item Adjacency pairs present
	\end{itemize}
	\begin{description}
		\item[G] You want me to climb in the tree?
		\item[\textcolor{red}{C}] [jɑː]
		\item[G] No way, silly monkey! You want to climb in the tree?
		\item[\textcolor{red}{C}] [jɑː]
	\end{description}
	
	 \begin{itemize}
		\item Mostly coherent, can use interjections/discourse markers
	\end{itemize}
	\begin{description}
		\item[\textcolor{red}{C}] oh, oh!
	\end{description}
	
\end{frame}

\subsubsection{Caretaker}
\begin{frame}
	\frametitle{Caretaker}
	
	\begin{itemize}
		\item Uses confirmation requests to prompt and scaffold the child.
		\item Supports interactionist theories
	\end{itemize}
	\begin{description}
		\item[G] That's a lot of snails, isn't it?
		\item[G] You put the snail in the garden, did you?
	\end{description}
	
\end{frame}

\section{Conclusion}
%Conclusion%
\begin{frame}
	\frametitle{Conclusion}
	
	\begin{itemize}
		\item Well into telegraphic stage $\rightarrow$ about 5 months ahead of expected development
		\item Evident by emerging subsystems
		\item Mostly supports behaviourist and interactionist theories
		\item Bella will continue to refine her language, approaching adult-like grammatical ability
	\end{itemize}
	
\end{frame}

\begin{frame}
	\frametitle{Conclusion}
	
	\begin{displayquote}
		In sum, then, "thinking about thinking" has to be a principal ingredient of any empowering practice of education.” - Jerome Bruner	
	\end{displayquote}
	
\end{frame}

\begin{frame}
	\centering
	Thank you for listening!
\end{frame}

\end{document}
